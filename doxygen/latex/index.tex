\hypertarget{index_Welcome}{}\section{Welcome}\label{index_Welcome}
Welcome to the Kodo A\-P\-I documentation. On these pages you will find the automatically generated Doxygen documentation for the Kodo project. We are using Doxygen mainly to create an overview of the functionality and A\-P\-Is available in Kodo.\hypertarget{index_kodo_arch}{}\subsection{Kodo Architecture}\label{index_kodo_arch}
Kodo is based on a special C++ design technique know an \char`\"{}mixin-\/layers\char`\"{} or \char`\"{}parametrized inheritance\char`\"{} if wish to modify or extend the functionality of Kodo you should first take a look at our manual which describes in more detail how Kodo's architecture is structured.


\begin{DoxyItemize}
\item \href{https://kodo.readthedocs.org/en/latest/}{\tt Kodo Manual}
\end{DoxyItemize}

You can find or documentation for the A\-P\-I Layers here, or under the \char`\"{}\-Modules\char`\"{} section available from either the lefthand side navigation pane or the top navigation pane.


\begin{DoxyItemize}
\item \hyperlink{group__api__layer}{A\-P\-I Layers}
\end{DoxyItemize}\hypertarget{index_building_docs}{}\subsection{Building the Documentation}\label{index_building_docs}
The Doxygen documentation for Kodo is located in the \char`\"{}doxygen\char`\"{} folder of the Kodo project. The A\-P\-I documentation can be built by running the doxygen command from within the doxygen folder this should generate the html version of the documentation into the \char`\"{}doxygen/html\char`\"{} folder. 